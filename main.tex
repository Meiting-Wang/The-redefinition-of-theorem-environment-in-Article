%使用xelatex编译
%-------导言区----------------
\documentclass[UTF8,space=auto]{ctexart} %UTF8编码,设置其处理空格的方式为auto
\usepackage[a4paper]{geometry} %设置纸张为A4大小
\usepackage{hyperref} %可使用超链接(含链接的跳转和一些命令)
\usepackage{amsmath} %最常用的数学宏包
\usepackage{amssymb} %包含更多的数学符号
\usepackage{lipsum} %英语假文宏包
\usepackage{ifthen} %使用\ifthenelse命令
\usepackage[dvipsnames]{xcolor}
\usepackage{fancyhdr} %设置页眉页脚的宏包

%超链接设置(依赖于hyperref宏包)
\hypersetup{
	pdftitle={这是展示在 PDF 信息中的标题},
	pdfauthor={王美庭},
	colorlinks=true,
	pdfsubject={这是 PDF 的主题,类似于摘要。},
	pdfkeywords={关键词1, 关键词2, ...},
	urlcolor={[RGB]{25, 128, 230}},%设置网页超链接的颜色
	linkcolor={[RGB]{25, 128, 230}},%设置目录、脚注、ref等超链接的颜色
	citecolor={[RGB]{25, 128, 230}},%设置cite超链接的颜色
}

%ctex宏包的设置
\ctexset{
	section = {
		format+ = {\raggedright} %节标题左对齐
	}
}


%页眉页脚的设置 (依赖于fancyhdr宏包)
\pagestyle{fancy} %使用fancy风格
\fancyhf{} %清除所有的页眉页脚
\fancyhead[L]{\slshape\leftmark} %页眉RE、LO位置章标题
\fancyhead[R]{\slshape\rightmark} %页眉LE、RO位置节标题
\fancyfoot[C]{\thepage} %页脚居中页码
\renewcommand{\headrulewidth}{0.4pt} %重定义页眉线宽度
%\renewcommand{\footrulewidth}{0.4pt} %重定义页脚线宽度

%--------------------定理类环境重定义-------------------
%-设置计数器
\newcounter{mycntthm} %设置新的计数器
\newcounter{mycntexam}
\setcounter{mycntthm}{0} %设定计数器默认值
\setcounter{mycntexam}{0}

%-开关设置
\newcommand{\thmseriesnamestyle}{Chinese} % Chinese 表示设置定理类名称为中文,English 表示设置定理类名称为英文
\newcommand{\thmseriesnumbering}{true} % true 表示对定理类环境进行编号,false 表示不对定理类环境进行编号

\ifthenelse{\equal{\thmseriesnamestyle}{Chinese}}{%
	\newcommand{\dfnname}{定义}
	\newcommand{\lemmaname}{引理}
	\newcommand{\thmname}{定理}
	\newcommand{\coroname}{推论}
	\renewcommand{\proofname}{证明}
	\newcommand{\examname}{例}
}%
{%
	\newcommand{\dfnname}{Definition}
	\newcommand{\lemmaname}{Lemma}
	\newcommand{\thmname}{Theorem}
	\newcommand{\coroname}{Corollary}
	\renewcommand{\proofname}{Proof}
	\newcommand{\examname}{Example}
}

%-自定义定理类环境
\newenvironment{dfn}[1][]%定义环境
{%
	\par%
	\bfseries%
	\dfnname%
	\ifthenelse{\equal{\thmseriesnumbering}{true}}{\refstepcounter{mycntthm}~\themycntthm}{}%
	\ifthenelse{\equal{#1}{}}{~~}{~(#1)~~}%
	\mdseries\itshape\color{black}%
}%
{\par}

\newenvironment{lemma}[1][]%引理环境
{%
	\par%
	\bfseries%
	\lemmaname%
	\ifthenelse{\equal{\thmseriesnumbering}{true}}{\refstepcounter{mycntthm}~\themycntthm}{}%
	\ifthenelse{\equal{#1}{}}{~~}{~(#1)~~}%
	\mdseries\itshape\color{black}%
}%
{\par}

\newenvironment{thm}[1][]%定理环境
{%
	\par%
	\bfseries%
	\thmname%
	\ifthenelse{\equal{\thmseriesnumbering}{true}}{\refstepcounter{mycntthm}~\themycntthm}{}%
	\ifthenelse{\equal{#1}{}}{~~}{~(#1)~~}%
	\mdseries\itshape\color{black}%
}%
{\par}

\newenvironment{coro}[1][]%推论环境
{%
	\par%
	\bfseries%
	\coroname%
	\ifthenelse{\equal{\thmseriesnumbering}{true}}{\refstepcounter{mycntthm}~\themycntthm}{}%
	\ifthenelse{\equal{#1}{}}{~~}{~(#1)~~}%
	\mdseries\itshape\color{black}%
}%
{\par}

\newenvironment{proof}[1][]%证明环境
{%
	\par%
	\bfseries%
	\proofname%
	\ifthenelse{\equal{#1}{}}{~~}{~(#1)~~}%
	\normalfont\color{black}%
}%
{\vspace{-\baselineskip}\hfill$\blacksquare$\par\vspace{\baselineskip}}

\newenvironment{exam}[1][]%例环境
{%
	\par%
	\bfseries%
	\examname%
	\ifthenelse{\equal{\thmseriesnumbering}{true}}{\refstepcounter{mycntexam}~\themycntexam}{}%
	\ifthenelse{\equal{#1}{}}{~~}{~(#1)~~}%
	\mdseries\itshape\color{black}%
}%
{\par}
 %导入定理类环境重定义文件(需要amssymb、ifthen宏包)

%标题页设置
\title{
	article 文档类中定理类环境的重定义
}
\author{王美庭\thanks{王美庭,暨南大学经济与社会研究院,Email: wangmeiting92@gmail.com}}
\date{\today}




%---------正文区-----------------
\begin{document}
\maketitle

\section{定义环境}

\lipsum[1][1-3]

\begin{dfn}\label{dfn:aa}%
	这是一个定义环境。This is a dfn env.
\end{dfn}

\begin{dfn}
	If 这是一个定义环境。This is a dfn env.
\end{dfn}

\begin{dfn}[xx 名词]
	这是一个定义环境。This is a dfn env.
\end{dfn}

\begin{dfn}[xx 名词]\label{dfn:bb}%
	If 这是一个定义环境。This is a dfn env.
\end{dfn}

\lipsum[1][1-3]

这是定义 \ref{dfn:aa},这是定义 \ref{dfn:bb}。


\section{引理环境}

\lipsum[1][1-3]

\begin{lemma}\label{dfn:cc}%
	这是一个引理环境。This is a lemma env.
\end{lemma}

\begin{lemma}
	If 这是一个引理环境。This is a lemma env.
\end{lemma}

\begin{lemma}[xx 引理]
	这是一个引理环境。This is a lemma env.
\end{lemma}

\begin{lemma}[xx 引理]\label{dfn:dd}%
	If 这是一个引理环境。This is a lemma env.
\end{lemma}

\lipsum[1][1-3]

这是引理 \ref{dfn:cc},这是引理 \ref{dfn:dd}。


\section{定理环境}

\lipsum[1][1-3]

\begin{thm}\label{dfn:ee}%
	这是一个定理环境。This is a thm env.
\end{thm}

\begin{thm}
	If 这是一个定理环境。This is a thm env.
\end{thm}

\begin{thm}[xx 定理]
	这是一个定理环境。This is a thm env.
\end{thm}

\begin{thm}[xx 定理]\label{dfn:ff}%
	If 这是一个定理环境。This is a thm env.
\end{thm}

\lipsum[1][1-3]

这是定理 \ref{dfn:ee},这是定理 \ref{dfn:ff}。


\section{推论环境}

\lipsum[1][1-3]

\begin{coro}\label{dfn:gg}%
	这是一个推论环境。This is a coro env.
\end{coro}

\begin{coro}
	If 这是一个推论环境。This is a coro env.
\end{coro}

\begin{coro}[xx 推论]
	这是一个推论环境。This is a coro env.
\end{coro}

\begin{coro}[xx 推论]\label{dfn:hh}%
	If 这是一个推论环境。This is a coro env.
\end{coro}

\lipsum[1][1-3]

这是推论 \ref{dfn:gg},这是推论 \ref{dfn:hh}。


\section{证明环境}
\lipsum[1][1-3]
\begin{proof}
	这是一个证明环境。This is a proof env.
\end{proof}

\begin{proof}[xx 定律]
	这是一个证明环境。This is a proof env.
\end{proof}

\lipsum[1][1-3]


\section{例环境}
\lipsum[1][1-3]

\begin{exam}\label{exam:aa}%
	这是一个例环境。This is a exam env.
\end{exam}

\begin{exam}
	If 这是一个例环境。This is a exam env.
\end{exam}

\begin{exam}[勾股定理的应用]
	这是一个例环境。This is a exam env.
\end{exam}

\begin{exam}[勾股定理的应用]\label{exam:bb}%
	If 这是一个例环境。This is a exam env.
\end{exam}

\lipsum[1][1-3]

这是例 \ref{exam:aa},这是例 \ref{exam:bb}。

\end{document}
